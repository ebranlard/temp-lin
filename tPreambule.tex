\usepackage{ifpdf}
\ifpdf
	\usepackage{graphicx}
	\usepackage{epstopdf} %don't forget the shell-escape command for pdflatex
    \usepackage[colorlinks,bookmarksopen,bookmarksnumbered,citecolor=red,urlcolor=red]{hyperref}
\else
	\usepackage{graphicx}
    \usepackage[dvips,colorlinks,bookmarksopen,bookmarksnumbered,citecolor=red,urlcolor=red]{hyperref}
\fi
\usepackage{type1cm}
% \usepackage{fullpage}
\usepackage{amsmath}
\usepackage{amssymb}
\usepackage{amsthm}
\usepackage{stmaryrd} %[| |] \llbracket
\usepackage{mhsetup}
\usepackage{mathtools}
\usepackage{relsize}
\usepackage{color} 
\usepackage{cases}  
\usepackage{fancybox}
% \usepackage{subfig}  % does not work with jpconf
\usepackage{subcaption}  % does not work with jpconf
\usepackage[table]{xcolor}   %lots of tools for colors
\usepackage{arydshln} % dash line in arrays: ``:'' in table cols, \hdashline and \cdashline
\usepackage{blkarray} % for block arrays (e.g. labelling matrices)
\usepackage{import}
\usepackage{listings} % for  lstlistings

\hypersetup{
    colorlinks = false,
    linkcolor =[rgb]{0.2,0.2,0.2},
    linkbordercolor =[rgb]{0.7,0.7,0.7}
}

% --------------------------------------------------------------------------------}
% --- CODE 
% --------------------------------------------------------------------------------{
\definecolor{mygreen}{rgb}{0,0.6,0}
\definecolor{myblue}{HTML}{00007b}
\definecolor{mygray}{rgb}{0.5,0.5,0.5}
\definecolor{mymauve}{rgb}{0.58,0,0.82}
\definecolor{myred}{HTML}{7b0000}

\renewcommand{\tt}[1]{\texttt{#1}}

% TODO TODO TOD, setting from my book are better
\lstset{ %
  basicstyle=\footnotesize\ttfamily,        % the size of the fonts that are used for the code
  breakatwhitespace=true,          % sets if automatic breaks should only happen at whitespace
  breaklines=true,                 % sets automatic line breaking
  extendedchars=true,              % lets you use non-ASCII characters; for 8-bits encodings only, does not work with UTF-8
  frame=single,	                   % adds a frame around the code
  keepspaces=true,                 % keeps spaces in text, useful for keeping indentation of code (possibly needs columns=flexible)
  language=Pascal,                 % the language of the code
  showspaces=false,                % show spaces everywhere adding particular underscores; it overrides 'showstringspaces'
  showstringspaces=false,          % underline spaces within strings only
  showtabs=false,                  % show tabs within strings adding particular underscores
  tabsize=2,	                   % sets default tabsize to 2 spaces
  backgroundcolor=\color{white},   % choose the background color; you must add \usepackage{color} or \usepackage{xcolor}; should come as last argument
  commentstyle=\color{mygreen},    % comment style
  keywordstyle=\bf\color{myblue},       % keyword style
  numberstyle=\tiny\color{mygray}, % the style that is used for the line-numbers
  stringstyle=\color{mymauve},     % string literal style
  basicstyle=\footnotesize\ttfamily, % tiny,scriptsize,footnotesize,small
}

\newcommand{\pascal}{%
\lstset{%
    language=pascal,
    morekeywords={unit,uses,raise,try,except,implementation},
    morecomment=[l]{//}}%
}
\newcommand{\matlab}{%
\lstset{%
    language=matlab,
    morekeywords={interp1} }
}
\newcommand{\fortran}{%
\lstset{%
    language=fortran,%
    literate= {\%}{{{\color{myblue}\%}}}1,
    morekeywords={interp1},
    keywordstyle = [2]{\color{myred}}, % teal
    morekeywords = [2]{CtrlOffset},
}
}
%    literate={{(}{{\CodeSymbol{(}}}},
% \lstdefinestyle{mystyle}
% {
%     language = C++,
%     basicstyle = {\ttfamily \color{main-color}},
%     backgroundcolor = {\color{back-color}},
%     stringstyle = {\color{string-color}},
%     keywordstyle = {\color{key-color}},
%     keywordstyle = [2]{\color{lime}},
%     keywordstyle = [3]{\color{yellow}},
%     keywordstyle = [4]{\color{teal}},
%     otherkeywords = {;,<<,>>,++},
%     morekeywords = [2]{;},
%     morekeywords = [3]{<<, >>},
%     morekeywords = [4]{++},
% }
% 

% --------------------------------------------------------------------------------}
% --- FIGURES 
% --------------------------------------------------------------------------------{
\DeclareGraphicsExtensions{.pdf,.PDF,.png,.jpg,.PNG,.JPG,.eps,.EPS,.Eps}
\graphicspath{{figs/}{figs_svg/}{figs_raw/}}
% --- svgtex
\newcommand{\svgtex}[4]{\begin{figure}[!htbp]%
 \centering%
 \def\svgwidth{#3\columnwidth}% used to be a scalebox
 \scalebox{#4}{\import{figs_svgtex/}{#1}}%
 \caption{#2}\label{fig:#1}%
 \end{figure}%
}
% --- svg
\newcommand{\svg}[3]{\begin{figure}[!htbp]
 \centering%
 \includegraphics[width=#3\textwidth]{#1}
 \caption{#2}\label{fig:#1}
 \end{figure}
}
% --- fig
\newcommand{\fig}[3]{\begin{figure}[!htbp]
 \centering%
 \includegraphics[width=#3\textwidth]{#1}
 \caption{#2}\label{fig:#1}
 \end{figure}
}

%%%%%%%%%%%%%%%%%%%%%%%%%%%%%%%%%%%%%%%%%%%%%%%%%%%%%%%%%
%%% Ensembles
%%%%%%%%%%%%%%%%%%%%%%%%%%%%%%%%%%%%%%%%%%%%%%%%%%%%%%%%%
\newcommand{\zz}{\emptyset}
\newcommand{\0}{\emptyset}

%%%%%%%%%%%%%%%%%%%%%%%%%%%%%%%%%%%%%%%%%%%%%%%%%%%%%%%%%
%%% vectors and matrix
%%%%%%%%%%%%%%%%%%%%%%%%%%%%%%%%%%%%%%%%%%%%%%%%%%%%%%%%%
%\renewcommand{\v}[1]{\overrightarrow{ \boldsymbol{ #1 } }}
% \renewcommand{\v}[1]{\underline{ #1 } }
%\renewcommand{\v}[1]{\overrightarrow{  #1  }}
%	\mathbf{#1} \boldsymbol{#1}
\renewcommand{\v}[1]{\boldsymbol{#1}}
\newcommand{\vv}[1]{\underline{ #1 } }
% \newcommand{\m}[1]{\underline{\underline{\boldsymbol{#1}}}}
% \newcommand{\m}[1]{\underline{\underline{\boldsymbol{#1}}}}
\newcommand{\mm}[1]{\boldsymbol{#1}}
\newcommand{\m}[1]{\boldsymbol{#1}}
\newcommand{\mrm}[1]{\boldsymbol{\mathrm{#1}}}
\def\identit{\hspace{0.08em}\makebox[5pt][c]{l\hspace{-0.55em}1}}
%\newcommand{\identit}{\ensuremath{\mathbbm{1}}}

\newcommand{\insys}[2]{\left.{#2}\right|_{{#1}}}  % in system 1

%%%%%%%%%%%%%%%%%%%%%%%%%%%%%%%%%%%%%%%%%%%%%%%%%%%%%%%%%
%%% Operators
%%%%%%%%%%%%%%%%%%%%%%%%%%%%%%%%%%%%%%%%%%%%%%%%%%%%%%%%%
%\DeclareMathOperator{\esssup}{ess\,sup}
\DeclareMathOperator{\atan}{atan}
\DeclareMathOperator{\asin}{asin}
\DeclareMathOperator{\acos}{acos}
\DeclareMathOperator{\sinc}{sinc}
% \DeclareMathOperator{\tr}{tr}
\renewcommand{\Re}{\operatorname{Re}}
\renewcommand{\Im}{\operatorname{Im}}
\DeclareMathOperator{\fft}{fft}
% \renewcommand{\d}{\mathrm{d}}
\renewcommand{\d}{\mathrm{d}}
\newcommand{\dr}{\d{r}}
\newcommand{\dm}{\d{m}}
\newcommand{\dt}{\d{t}}
\newcommand{\dA}{\d{A}}
\newcommand{\dQ}{\d{Q}}
\newcommand{\dT}{\d{T}}
\newcommand{\dP}{\d{P}}
% \newcommand{\stimes}{{\times}}
% \newcommand{\scdot}{{\cdot}}

%%%%%%%%%%%%%%%%%%%%%%%%%%%%%%%%%%%%%%%%%%%%%%%%%%%%%%%%%
%%% Wind energy notations 
%%%%%%%%%%%%%%%%%%%%%%%%%%%%%%%%%%%%%%%%%%%%%%%%%%%%%%%%%
\newcommand{\MT}{\ensuremath{\text{MT}}}
\newcommand{\BET}{\ensuremath{\text{BT}}}
\newcommand{\KJ}{\ensuremath{\text{KJ}}}
\newcommand{\LOD}{\upvarepsilon_{\mathsmaller{l\!/\!d}}}
\newcommand{\CTloc}{\ensuremath{C_{T_r}}}
\newcommand{\CPloc}{\ensuremath{C_{P_r}}}
\newcommand{\CQloc}{\ensuremath{C_{Q_r}}}

%%%%%%%%%%%%%%%%%%%%%%%%%%%%%%%%%%%%%%%%%%%%%%%%%%%%%%%%%
%%% Meca flu
%%%%%%%%%%%%%%%%%%%%%%%%%%%%%%%%%%%%%%%%%%%%%%%%%%%%%%%%%
\newcommand{\up}{\v{u}^{\!\mathsmaller{+}}} %fluid velocity
\newcommand{\um}{\v{u}^{\!\mathsmaller{-}}} %fluid velocity
\newcommand{\V}{\v{V}} %fluid velocity
\newcommand{\om}{\v{\omega}} %fluid velocity
\newcommand{\ve}{\v{v}} %fluid velocity
\newcommand{\D}{{\Omega(t)}} %domain
\newcommand{\dD}{{\partial\Omega(t)}} %frontier
\newcommand{\dl}{\,{\d}l}
\newcommand{\dS}{\,{\d}S}
\newcommand{\dv}{\,{\d}v}
\DeclareMathOperator{\grad}{grad}
\DeclareMathOperator{\gradv}{\grad}
\newcommand{\gradd}[1]{\v{\grad}\,#1}
\DeclareMathOperator{\rot}{curl}
\newcommand{\nab}{\nabla\!}
\DeclareMathOperator{\nabb}{\underline{\underline{\nabla}}}
\renewcommand{\div}{\operatorname{div}}
\newcommand{\dpart}[1]{\frac{d#1}{dt}}
\newcommand{\dfix}[1]{\frac{\partial #1}{\partial t}}
\newcommand{\dmob}[2]{\left.\frac{\delta #2}{\delta t}\right|_{#1}}
\newcommand{\dpx}[1]{\frac{\partial#1}{\partial x}}
\newcommand{\dpy}[1]{\frac{\partial#1}{\partial y}}
\newcommand{\dpz}[1]{\frac{\partial#1}{\partial z}}
\newcommand{\dpr}[1]{\frac{\partial#1}{\partial r}}
\newcommand{\dpt}[1]{\frac{\partial#1}{\partial \theta}}
\newcommand{\disc}[1]{\left\llbracket#1\right\rrbracket}

%%%%%%%%%%%%%%%%%%%%%%%%%%%%%%%%%%%%%%%%%%%%%%%%%%%%%%%%%
%%% Relations
%%%%%%%%%%%%%%%%%%%%%%%%%%%%%%%%%%%%%%%%%%%%%%%%%%%%%%%%%
\newcommand\overArrow[2][=]{\stackrel{\overset{\makebox[0pt]{\text{#2}}}{\uparrow}}{#1}}
\renewcommand{\eqref}[1]{\overArrow[=]{\scriptsize\ref{#1}}} 
\newcommand{\eqtext}[1]{\overArrow[=]{\scriptsize\text{#1}}} 
%\newcommand{\eqdef}{\stackrel{\mathrm{def}}{=}}
% \newcommand{\eqdef}{  \stackrel{\wedge}{=}} %equal per definition
\newcommand{\eqdef}{\stackrel{\mathsmaller{\mathsmaller{\mathsmaller{\triangle}}}}{=}} 
\newcommand{\eqapprox}{  \stackrel{\sim}{=}} %equal per definition
\newcommand{\equivalent}[1]{\underset{#1}{\sim}} %equivalence at #1
\newcommand{\tends}[1]{\xrightarrow[#1]{}}  %tends to, when #1


%%%%%%%%%%%%%%%%%%%%%%%%%%%%%%%%%%%%%%%%%%%%%%%%%%%%%%%%%
%%% fractions
%%%%%%%%%%%%%%%%%%%%%%%%%%%%%%%%%%%%%%%%%%%%%%%%%%%%%%%%%
\renewcommand{\dfrac}[2]{{\displaystyle \frac{#1}{#2}}}
%\renewcommand{\tfrac}[2]{{\ensuremath{\textstyle \frac{#1}{#
\renewcommand{\tfrac}[2]{{\ensuremath{#1/#2}}}


%%%%%%%%%%%%%%%%%%%%%%%%%%%%%%%%%%%%%%%%%%%%%%%%%%%%%%%%%
%%% Norms
%%%%%%%%%%%%%%%%%%%%%%%%%%%%%%%%%%%%%%%%%%%%%%%%%%%%%%%%%
\providecommand{\abs}[1]{\lvert#1\rvert}
\providecommand{\norm}[1]{\lVert#1\rVert}

% --------------------------------------------------------------------------------}
% ---  
% --------------------------------------------------------------------------------{
% Matrices
\newcommand{\M} {{\m{M}}}
\newcommand{\Mrm} {{\m{\mathrm{M}}}}
\newcommand{\Mrmdot} {{\m{\dot{\mathrm{M}}}}}
\newcommand{\K} {{\m{K}}}
\newcommand{\Jt}{\m{B}_\theta}
\newcommand{\Jx}{\m{B}_x}
\newcommand{\Jxhat}{\m{\hat{B}}_x}
\newcommand{\Jthat}{\m{\hat{B}}_\theta}
\newcommand{\Jxtil}{\m{\tilde{B}}_x}
\newcommand{\Jttil}{\m{\tilde{B}}_\theta}
\newcommand{\J} {\mrm{B}}
\newcommand{\Jdot}{\mrm{\dot{B}}}
\newcommand{\R}{\m{R}}
\newcommand{\Rrm}{\m{\mathrm{R}}}
% Vector Variables
\renewcommand{\r}{\v{r}}
\renewcommand{\u}{\v{u}}
\newcommand{\q}{\v{\mathrm{q}}}
\newcommand{\g}{\v{g}}
\newcommand{\U}{\v{U}}
\newcommand{\n}{\v{n}}
\newcommand{\x}{\v{x}}
\newcommand{\xrm}  {\v{\mathrm{x}}}
\newcommand{\vrm}  {\v{\mathrm{v}}}
\newcommand{\vrmdot}{\v{\dot{\mathrm{v}}}}
\newcommand{\arm}  {\v{\mathrm{a}}}
\newcommand{\frm}  {\v{\mathrm{f}}}
\newcommand{\qdot} {\v{\dot{\mathrm{q}}}}
\newcommand{\gdot} {\v{\dot{g}}}
\newcommand{\xdot} {\v{\dot{x}}}
\newcommand{\rdot} {\v{\dot{r}}}
\newcommand{\udot} {\v{\dot{u}}}
\newcommand{\vdot} {\v{\dot{v}}}
\newcommand{\omdot}{\v{\dot{\omega}}}
\newcommand{\xrmdot}{\v{\dot{\mathrm{x}}}}
\newcommand{\qddot}{\v{\ddot{\mathrm{q}}}}
\newcommand{\gddot}{\v{\ddot{g}}}
\newcommand{\xddot}{\v{\ddot{x}}}
\newcommand{\uddot}{\v{\ddot{u}}}
\newcommand{\vddot}{\v{\ddot{v}}}
\newcommand{\rddot}{\v{\ddot{r}}}
\newcommand{\xrmddot}{\v{\ddot{\mathrm{x}}}}
\renewcommand{\r}{\v{r}}
\newcommand{\s}{\v{s}}
\newcommand{\ddt}[1]{\frac{d#1}{dt}}
\newcommand{\ddx}[1]{\frac{d#1}{dx}}
\newcommand{\mphi}{\m{\Phi}}
% \newcommand{\om}{\v{\omega}}
\newcommand{\omtil} {\m{\tilde{\omega}}}
\newcommand{\omptil}{\m{\tilde{\omega'}}}
\newcommand{\rtil}  {\m{\tilde{r}}}
\newcommand{\stil}  {\m{\tilde{s}}}
\newcommand{\rhotil}{\m{\tilde{\rho}}}
\newcommand{\rhoptil}{\m{\tilde{{\rho'}}}}




%%%%%%%%%%%%%%%%%%%%%%%%%%%%%%%%%%%%%%%%%%%%%%%%%%%%%%%%%
%%% Misc
%%%%%%%%%%%%%%%%%%%%%%%%%%%%%%%%%%%%%%%%%%%%%%%%%%%%%%%%%
\newcommand{\lbar}{\overline}
\renewcommand{\bar}{\overline}
\renewcommand{\deg}{\ensuremath{^\circ}}
\renewcommand{\dim}[1]{\rlap{\ensuremath{\qquad\text{[ #1 ]}}}}
\newcommand{\cst}{\text{cst}}
\newcommand{\AR}{\ensuremath{\text{A\!R}}}
\mathchardef\mhyphen="2D   % Needs amsmath


\newcommand{\omAB}[2]{\v{\omega}\left(\frac{#1}{#2}\right)}

\newcommand{\Fg}{\ensuremath{\v{F\!_g}}}
\newcommand{\GF}{\ensuremath{\v{\mathrm{G}\!\mathrm{F}}}}
\newcommand{\GM}{\ensuremath{   \mathrm{G}\!\mathrm{M}}}
\newcommand{\GO}{\ensuremath{   \mathrm{G}\!\mathrm{\omega}}}
\newcommand{\GXI}{\ensuremath{  \mathrm{G}\!\mathrm{\xi}}}
\newcommand{\GD}{\ensuremath{   \mathrm{G}\!\mathrm{D}}}
\newcommand{\GK}{\ensuremath{   \mathrm{G}\!\mathrm{K}}}
\newcommand{\GZ}{\ensuremath{   \mathrm{G}\!\mathrm{Z}}}
\newcommand{\GX}{\ensuremath{   \mathrm{G}\!\mathrm{X}}}
\newcommand{\Xg}{\ensuremath{\v{X\!_g}}}
\newcommand{\dXg}{\ensuremath{\v{\dot{X\!_g}}}}
\newcommand{\ddXg}{\ensuremath{\v{\ddot{X\!_g}}}}


\newcommand{\weird}[1]{{\color{red}{!!!#1!!!}}}
\newcommand{\myquote}[1]{Quote: {\color{red}{#1}}}
\newcommand{\todo}[1]{{\colorbox{yellow}{TODO:}} #1\colorbox{yellow}{/}}

\newcommand\runinhead[1]{\par\vspace{0.3cm}\noindent\textbf{#1}\ }%

\setcounter{tocdepth}{2}     % Profondeur table des matiere
% \setcounter{tocdepth}{3}
% \linespread{2}\selectfont

% --------------------------------------------------------------------------------}
% --- Spacing 
% --------------------------------------------------------------------------------{
\providecommand{\tightlist}{\setlength{\itemsep}{0pt}\setlength{\parskip}{0pt}}
% --------------------------------------------------------------------------------}
% --- INDEX  
% --------------------------------------------------------------------------------{
\newcommand{\indexdef}[1]{\textit{#1}\index{#1}}%
\newcommand{\indexme}[1]{\textit{#1}\index{#1}}% same as indexdef
